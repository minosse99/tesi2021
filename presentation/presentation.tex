\documentclass[xcolor=dvipsnames]{beamer}
\usepackage{graphicx} 
\usepackage{booktabs} 
\mode<presentation>{
    \usetheme{madrid}
    \definecolor{mygreen}{rgb}{0, 0.75, 0} % UBC Blue (primary)
    \usecolortheme[named=mygreen]{structure}
}

\title{STRUMENTI DI ORCHESTRAZIONE E ANALISI DI WORKFLOW NEL MACHINE LEARNING}
\subtitle[]{Analisi Case Study e illustrazione Framework di Orchestrazione di pipeline di lavoro}
\author{Simone Boldrini}
\date{13 Ottobre 2021}
\institute[Unibo]{Alma Mater Studiorum - Universitá di Bologna \\ Facoltá di Scienze}

\begin{document}

\maketitle

\AtBeginSection[]
{
    \begin{frame}
    \setbeamercolor{background canvas}{bg=green} 
    \frametitle{Table of Contents}
    \tableofcontents[currentsection]
    \end{frame}
}
\section{Casi di Studio}
\begin{frame}
    
\frametitle{Casi di studio1}
Contents of the first slide
\end{frame}
\subsection{MLN}
\begin{frame}
\frametitle{Second Slide}
Contents of the second slide
\end{frame}
\subsection{Insider Threat Detection}
\begin{frame}
    \frametitle{Raccolta Dati}
    Contents of the second slide
\end{frame}
\section{Analisi}
\begin{frame}
    \frametitle{Individuazione del Problema}
    Contents of the second slide
\end{frame}
\begin{frame}
    \frametitle{Elaborazione del Modello}
    Contents of the second slide
\end{frame}
\begin{frame}
    \frametitle{Conclusioni}
    Contents of the second slide
\end{frame}
\section{Framework}

\end{document}
