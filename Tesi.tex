\documentclass[12pt,a4paper]{report}
\usepackage[italian]{babel}
\usepackage{newlfont}
\usepackage{amsmath}
\usepackage{graphicx}
\usepackage{amsfonts}
\usepackage[automake]{glossaries-extra}
\usepackage{subfiles}
\usepackage{biblatex}
\addbibresource {Bibliography.bib}

\graphicspath{ {./image/} }

\makeglossaries

\newglossaryentry{framework}{name={Framework},description={architettura di supporto sulla quale un software puó essere progettato e realizzato, facilitandone lo sviluppo}}

\newcommand{\chapquote}[3]{\begin{quotation} \textit{#1} \end{quotation} \begin{flushright} - #2, \textit{#3}\end{flushright} }
\textwidth=450pt\oddsidemargin=0pt
\begin{document}

\subfile{chapter/frontespizio}

%blank page instruction ------
\begingroup
  \pagestyle{empty}
  \null
  \newpage  
\endgroup

\tableofcontents

%Chapter 1: Introduction
\subfile{chapter/introduzione}

%Chapter 2: Case Study
\subfile{chapter/casestudy}

%Chapter 3: CS' Analysis
\subfile{chapter/analisi}

%Chapter 4: Orchestration of WF
\subfile{chapter/framework}

\printglossaries

\printbibliography

\end{document}